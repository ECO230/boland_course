% Options for packages loaded elsewhere
% Options for packages loaded elsewhere
\PassOptionsToPackage{unicode}{hyperref}
\PassOptionsToPackage{hyphens}{url}
\PassOptionsToPackage{dvipsnames,svgnames,x11names}{xcolor}
%
\documentclass[
  letterpaper,
  DIV=11,
  numbers=noendperiod]{scrartcl}
\usepackage{xcolor}
\usepackage[margin=1in]{geometry}
\usepackage{amsmath,amssymb}
\setcounter{secnumdepth}{-\maxdimen} % remove section numbering
\usepackage{iftex}
\ifPDFTeX
  \usepackage[T1]{fontenc}
  \usepackage[utf8]{inputenc}
  \usepackage{textcomp} % provide euro and other symbols
\else % if luatex or xetex
  \usepackage{unicode-math} % this also loads fontspec
  \defaultfontfeatures{Scale=MatchLowercase}
  \defaultfontfeatures[\rmfamily]{Ligatures=TeX,Scale=1}
\fi
\usepackage{lmodern}
\ifPDFTeX\else
  % xetex/luatex font selection
\fi
% Use upquote if available, for straight quotes in verbatim environments
\IfFileExists{upquote.sty}{\usepackage{upquote}}{}
\IfFileExists{microtype.sty}{% use microtype if available
  \usepackage[]{microtype}
  \UseMicrotypeSet[protrusion]{basicmath} % disable protrusion for tt fonts
}{}
\makeatletter
\@ifundefined{KOMAClassName}{% if non-KOMA class
  \IfFileExists{parskip.sty}{%
    \usepackage{parskip}
  }{% else
    \setlength{\parindent}{0pt}
    \setlength{\parskip}{6pt plus 2pt minus 1pt}}
}{% if KOMA class
  \KOMAoptions{parskip=half}}
\makeatother
% Make \paragraph and \subparagraph free-standing
\makeatletter
\ifx\paragraph\undefined\else
  \let\oldparagraph\paragraph
  \renewcommand{\paragraph}{
    \@ifstar
      \xxxParagraphStar
      \xxxParagraphNoStar
  }
  \newcommand{\xxxParagraphStar}[1]{\oldparagraph*{#1}\mbox{}}
  \newcommand{\xxxParagraphNoStar}[1]{\oldparagraph{#1}\mbox{}}
\fi
\ifx\subparagraph\undefined\else
  \let\oldsubparagraph\subparagraph
  \renewcommand{\subparagraph}{
    \@ifstar
      \xxxSubParagraphStar
      \xxxSubParagraphNoStar
  }
  \newcommand{\xxxSubParagraphStar}[1]{\oldsubparagraph*{#1}\mbox{}}
  \newcommand{\xxxSubParagraphNoStar}[1]{\oldsubparagraph{#1}\mbox{}}
\fi
\makeatother


\usepackage{longtable,booktabs,array}
\usepackage{calc} % for calculating minipage widths
% Correct order of tables after \paragraph or \subparagraph
\usepackage{etoolbox}
\makeatletter
\patchcmd\longtable{\par}{\if@noskipsec\mbox{}\fi\par}{}{}
\makeatother
% Allow footnotes in longtable head/foot
\IfFileExists{footnotehyper.sty}{\usepackage{footnotehyper}}{\usepackage{footnote}}
\makesavenoteenv{longtable}
\usepackage{graphicx}
\makeatletter
\newsavebox\pandoc@box
\newcommand*\pandocbounded[1]{% scales image to fit in text height/width
  \sbox\pandoc@box{#1}%
  \Gscale@div\@tempa{\textheight}{\dimexpr\ht\pandoc@box+\dp\pandoc@box\relax}%
  \Gscale@div\@tempb{\linewidth}{\wd\pandoc@box}%
  \ifdim\@tempb\p@<\@tempa\p@\let\@tempa\@tempb\fi% select the smaller of both
  \ifdim\@tempa\p@<\p@\scalebox{\@tempa}{\usebox\pandoc@box}%
  \else\usebox{\pandoc@box}%
  \fi%
}
% Set default figure placement to htbp
\def\fps@figure{htbp}
\makeatother





\setlength{\emergencystretch}{3em} % prevent overfull lines

\providecommand{\tightlist}{%
  \setlength{\itemsep}{0pt}\setlength{\parskip}{0pt}}



 


\usepackage{booktabs}
\usepackage{longtable}
\usepackage{array}
\usepackage{multirow}
\usepackage{wrapfig}
\usepackage{float}
\usepackage{colortbl}
\usepackage{pdflscape}
\usepackage{tabu}
\usepackage{threeparttable}
\usepackage{threeparttablex}
\usepackage[normalem]{ulem}
\usepackage{makecell}
\usepackage{xcolor}
\KOMAoption{captions}{tableheading}
\makeatletter
\@ifpackageloaded{tcolorbox}{}{\usepackage[skins,breakable]{tcolorbox}}
\@ifpackageloaded{fontawesome5}{}{\usepackage{fontawesome5}}
\definecolor{quarto-callout-color}{HTML}{909090}
\definecolor{quarto-callout-note-color}{HTML}{0758E5}
\definecolor{quarto-callout-important-color}{HTML}{CC1914}
\definecolor{quarto-callout-warning-color}{HTML}{EB9113}
\definecolor{quarto-callout-tip-color}{HTML}{00A047}
\definecolor{quarto-callout-caution-color}{HTML}{FC5300}
\definecolor{quarto-callout-color-frame}{HTML}{acacac}
\definecolor{quarto-callout-note-color-frame}{HTML}{4582ec}
\definecolor{quarto-callout-important-color-frame}{HTML}{d9534f}
\definecolor{quarto-callout-warning-color-frame}{HTML}{f0ad4e}
\definecolor{quarto-callout-tip-color-frame}{HTML}{02b875}
\definecolor{quarto-callout-caution-color-frame}{HTML}{fd7e14}
\makeatother
\makeatletter
\@ifpackageloaded{caption}{}{\usepackage{caption}}
\AtBeginDocument{%
\ifdefined\contentsname
  \renewcommand*\contentsname{Table of contents}
\else
  \newcommand\contentsname{Table of contents}
\fi
\ifdefined\listfigurename
  \renewcommand*\listfigurename{List of Figures}
\else
  \newcommand\listfigurename{List of Figures}
\fi
\ifdefined\listtablename
  \renewcommand*\listtablename{List of Tables}
\else
  \newcommand\listtablename{List of Tables}
\fi
\ifdefined\figurename
  \renewcommand*\figurename{Figure}
\else
  \newcommand\figurename{Figure}
\fi
\ifdefined\tablename
  \renewcommand*\tablename{Table}
\else
  \newcommand\tablename{Table}
\fi
}
\@ifpackageloaded{float}{}{\usepackage{float}}
\floatstyle{ruled}
\@ifundefined{c@chapter}{\newfloat{codelisting}{h}{lop}}{\newfloat{codelisting}{h}{lop}[chapter]}
\floatname{codelisting}{Listing}
\newcommand*\listoflistings{\listof{codelisting}{List of Listings}}
\makeatother
\makeatletter
\makeatother
\makeatletter
\@ifpackageloaded{caption}{}{\usepackage{caption}}
\@ifpackageloaded{subcaption}{}{\usepackage{subcaption}}
\makeatother
\usepackage{bookmark}
\IfFileExists{xurl.sty}{\usepackage{xurl}}{} % add URL line breaks if available
\urlstyle{same}
\hypersetup{
  pdftitle={ECO 230 Business and Economic Research \& Communication},
  colorlinks=true,
  linkcolor={blue},
  filecolor={Maroon},
  citecolor={Blue},
  urlcolor={Blue},
  pdfcreator={LaTeX via pandoc}}


\title{ECO 230 Business and Economic Research \& Communication}
\usepackage{etoolbox}
\makeatletter
\providecommand{\subtitle}[1]{% add subtitle to \maketitle
  \apptocmd{\@title}{\par {\large #1 \par}}{}{}
}
\makeatother
\subtitle{Spring 2026 · Section 12 · 3 Credits}
\author{}
\date{}
\begin{document}
\maketitle

\renewcommand*\contentsname{Table of contents}
{
\hypersetup{linkcolor=}
\setcounter{tocdepth}{3}
\tableofcontents
}

\subsection{Course information}\label{course-information}

\textbf{ECO 230 Business and Economic Research \& Communication}

\textbf{Spring 2026 \textbar{} Section 12 \textbar{} Credits: 3}

Meets in Wittich 112 Tuesday 5:30 pm to 8:15 pm.

\subsubsection{Instructor information}\label{instructor-information}

\begin{itemize}
\tightlist
\item
  Mike Boland, MBA\\
\item
  Office Location: Non Extant\\
\item
  Student Hours: Mon.--Fri. Noon--1pm in Zoom or by appointment\\
\item
  Class Webpage: \url{https://www.uwlax.edu/canvas}\\
\item
  Email:
  \href{mailto:mboland@uwlax.edu}{\nolinkurl{mboland@uwlax.edu}}\\
\item
  Cell Phone: 608-385-6497
\end{itemize}

\subsubsection{Final exam}\label{final-exam}

\textbf{\emph{Friday May 12th 7:00 pm -- 9:00 pm in Wittich 112}}

\subsection{Course description}\label{course-description}

Building on the foundation in statistics acquired in STAT 145, students
will continue to develop and will apply skills in data analysis to aid
in business decision making. These skills include data collection, data
summarization, data visualization, statistical inference, and
communication of data in business contexts. Students will learn and
apply best practices for research design and analysis to address
authentic business cases. Students will build these skills in
collaboration with each other and through engagement with business and
community leaders. The course also discusses effective survey design and
current privacy and ethical issues in collecting and using data.
Prerequisite: ENG 110 or ENG 112; STAT 145; CBA major, CASSH economics
major, or healthcare analytics management minor. Offered Fall, Spring.

\subsubsection{Prerequisites}\label{prerequisites}

Prerequisite: ENG 110 or ENG 112; STAT 145; CBA major, CLS economics
major, or Healthcare Analytics Management Minor.

\subsection{Learning outcomes}\label{learning-outcomes}

\subsubsection{Mapped to CBA Undergraduate Curriculum Goals and
Objectives}\label{mapped-to-cba-undergraduate-curriculum-goals-and-objectives}

\begin{enumerate}
\def\labelenumi{\arabic{enumi}.}
\tightlist
\item
  Develop the ability to articulate a business problem or opportunity
  using qualitative and quantitative evidence and propose an analysis
  plan to identify potential solutions, including collection of primary
  data via surveys. (CT-1)
\item
  Describe, summarize, and interpret data using descriptive statistics,
  inferential statistics, and data visualization.
\item
  Develop foundational skills related to spreadsheets, interactive data
  visualization software, and scripting languages used for data analysis
  and visualization.
\item
  Communicate the purpose, methods, and results of analysis to authentic
  audiences in appropriate written and oral formats. (CS-1, CS-2)
\item
  Apply research design and data analysis best practices to complete
  case-based projects and to critique others' analyses. (CT-1)
\item
  Describe current debates in research ethics and data privacy and the
  implications for business research. (SR-1)
\end{enumerate}

\subsubsection{CBA Undergraduate Curriculum Goals and Objectives
referenced
above}\label{cba-undergraduate-curriculum-goals-and-objectives-referenced-above}

\begin{itemize}
\tightlist
\item
  \textbf{Communication Goal}: Our students will be able to convey
  information and ideas effectively.

  \begin{itemize}
  \tightlist
  \item
    (CS-1) Learning Objective: Students will convey information and
    ideas in professional business reports\\
  \item
    (CS-2) Learning Objective: Students will convey information and
    ideas in oral presentations.
  \end{itemize}
\item
  \textbf{Decision Making and Critical Thinking Goal}: Our students will
  be able to think critically when evaluating decisions.

  \begin{itemize}
  \tightlist
  \item
    (CT-1) Learning Objective: Students will evaluate alternatives and
    understand the ramifications of those alternatives within a given
    business context.
  \end{itemize}
\item
  \textbf{Social Responsibility Goal}: Students will demonstrate the
  role of social responsibility in business decisions.

  \begin{itemize}
  \tightlist
  \item
    (SR-1) Learning Objective: Students will be able to identify and
    apply different frameworks of social responsibility to business
    problems and recognize the short- and long-term effects on
    stakeholders and society.
  \end{itemize}
\end{itemize}

Remaining CBA Goals and Objectives along with rubrics for each can be
found here:\\
\url{https://www.uwlax.edu/cba/undergraduate-curriculum-learning-goals-and-objectives2/\#tmcommunication-goals}

\subsection{Materials \& tools}\label{materials-tools}

There is no official textbook for this course. All readings will be
provided electronically via Canvas. We will also use the following
software and instructional tools which are free of charge to you due to
your enrollment in this course. Instructions for use will be provided in
class and on Canvas as needed.

\begin{itemize}
\tightlist
\item
  \textbf{Excel}\\
\item
  \textbf{Tableau Desktop}\\
\item
  \textbf{Posit Cloud}\\
\item
  \textbf{Qualtrics}\\
\item
  \textbf{CATME}
\end{itemize}

\subsection{Format}\label{format}

This is a face-to-face course. You may be asked to reference materials
or participate online through the learning management system, Canvas. If
that is the case, you will need your UWL NetID to login to the course
from the Canvas homepage \url{http://www.uwlax.edu/canvas}.

\subsection{Grading}\label{grading}

\subsubsection{Calculations}\label{calculations}

Your overall grade consists of the following assessments, arranged by
type and value. Your final grade can be computed by multiplying the
scores received by the percentages reported below to arrive at a
weighted average. Team assignments are denoted by (T) but please note
grades received for team project deliverables will be adjusted by your
Peer Evaluation score.

\begin{longtable*}[t]{lll}
\toprule
Assignment & \# & Total \% Final Grade\\
\midrule
Homework, Lab Prep \& In Class Labs & \textasciitilde{}10 & 25\%\\
Quizzes & \textasciitilde{}5 & 15\%\\
Final Exam \& Practicum & 1 & 20\%\\
Case Projects (T) & 2 & 30\%\\
Participation / Discussion &  & 10\%\\
Total &  & 100\%\\
\bottomrule
\end{longtable*}

\subsubsection{Calculation of individual grades for team project
work}\label{calculation-of-individual-grades-for-team-project-work}

All work for the assignments denoted as Team Projects by (T) above will
be scored out of 100 points to assign a Team Grade. Individual grades
will be computed by multiplying the Team Grade by the CATME peer
evaluation score you receive for that work. The CATME peer evaluation is
a psychometrically validated and behaviorally anchored rating scale
developed under a National Science Foundation Grant and used by
universities throughout the country. Through a secure web-based
platform, you will rate your own and your teammate's performance on each
project. This information will be reported in the aggregate to your
teammates (they cannot see who gave them what scores) and will produce a
Peer Evaluation Score. Most students receive a score of 1, those that
exhibit exceptional performance receive scores above 1 with a maximum of
1.05. Underperforming teammates receive scores below 1, and as low as
0.4.

\begin{tcolorbox}[enhanced jigsaw, breakable, coltitle=black, opacitybacktitle=0.6, colframe=quarto-callout-note-color-frame, left=2mm, leftrule=.75mm, title=\textcolor{quarto-callout-note-color}{\faInfo}\hspace{0.5em}{Note}, rightrule=.15mm, titlerule=0mm, opacityback=0, arc=.35mm, toprule=.15mm, colback=white, bottomtitle=1mm, bottomrule=.15mm, colbacktitle=quarto-callout-note-color!10!white, toptitle=1mm]

\textbf{Note:} This is not a ``zero sum game''. It is possible for team
members to get scores above 1 without others getting scores below. The
algorithm flags evaluations that appear to manipulate ratings to inflate
their own scores. Your score will automatically be reduced by 0.05 if
you fail to complete the Peer Evaluation on or before the due date.

\end{tcolorbox}

\subsubsection{Grievance procedure for peer evaluation
scores}\label{grievance-procedure-for-peer-evaluation-scores}

Poor peer evaluations can lead an individual grade to fall substantially
below the Team Grade. For example, a project that received a score of
95/100 would produce an individual grade of 38/100 if you receive a peer
evaluation score of 0.4. If you feel the score you received does not
accurately reflect your contributions to Team Project work, you have the
option to file a grievance. To do so you must follow this procedure:

\begin{enumerate}
\def\labelenumi{\arabic{enumi}.}
\tightlist
\item
  Within 2 business days of the official notification that grades have
  been posted, communicate to your instructor via email your intent to
  file a grievance. Official notifications of grade posting will appear
  on Canvas notifications.
\item
  Within 1 business day of notifying your intent, prepare and email a
  one page formal letter addressed to your teammates and including the
  following information:

  \begin{itemize}
  \tightlist
  \item
    \textbf{Your understanding of the work you had agreed to do} for the
    deliverable. Be sure to refer to any documentation that is relevant
    (Meeting Minutes, emails or group chats, revision histories in
    documents etc.). Be prepared to provide any documentation you
    mention in the letter if requested.
  \item
    \textbf{A summary of the work you actually did.} Refer to
    documentation as above.
  \item
    \textbf{Justification for any discrepancies between (a) and (b).} If
    this documentation includes any confidential health or personal
    matters you do not wish to disclose to your team, please raise this
    with me or contact Student Life Office for council. Note: In these
    cases it is best to reach out to me or Student Life before you are
    at the point of receiving a bad peer evaluation.
  \item
    \textbf{A statement of the peer evaluation score you believe you
    have earned.}
  \end{itemize}
\item
  If after reviewing your letter I believe the case has merit, I will
  send it on to your teammates and schedule an arbitration. If I
  determine no arbitration is warranted, I will schedule a one-on-one
  meeting with you to discuss.
\item
  If arbitration is warranted, it must be conducted in-person with all
  team members present. The only exception to this requirement would be
  a case where the timeline in the procedure would place the earliest
  possible arbitration date after the final exam date. In those cases, a
  virtual meeting may be agreed upon or incomplete grades may be
  assigned until an in-person meeting can be scheduled. I will serve as
  the arbitrator. You and your teammates will have the opportunity to
  present your case and reconcile evidence referenced in your letter. At
  the end of the meeting, all team members (including yourself) will
  submit the score they believe you have earned to me via secret ballot.
  I will use this information in making my final determination, but the
  final determination is mine alone.
\end{enumerate}

\subsubsection{Late assignments, missed
quizzes}\label{late-assignments-missed-quizzes}

\paragraph{Individual assignments}\label{individual-assignments}

\begin{tcolorbox}[enhanced jigsaw, breakable, coltitle=black, opacitybacktitle=0.6, colframe=quarto-callout-warning-color-frame, left=2mm, leftrule=.75mm, title=\textcolor{quarto-callout-warning-color}{\faExclamationTriangle}\hspace{0.5em}{Warning}, rightrule=.15mm, titlerule=0mm, opacityback=0, arc=.35mm, toprule=.15mm, colback=white, bottomtitle=1mm, bottomrule=.15mm, colbacktitle=quarto-callout-warning-color!10!white, toptitle=1mm]

Individual assignments are due on the dates indicated on the Canvas
Weekly Agendas. Late or missing assignments will not be graded.
\textbf{All assignments in a module must be completed before the Module
Quiz.} If extenuating circumstances impact your ability to meet
deadlines or participate in class activities, you are responsible for
alerting me as soon as possible. When conflicts are known ahead of the
deadline you must notify me ahead of the deadline. All communication
regarding missed or late individual work should be confirmed by email --
even if we speak in person or during student hours, you must follow up
via email to confirm so that we both have a record of the agreement.

\end{tcolorbox}

\paragraph{Team assignments}\label{team-assignments}

Deadlines for Team Assignments may be extended for the entire class if
there are extenuating circumstances or granted to teams if requested
with appropriate justification and all team members agree to the
request. \textbf{Presentations cannot be delayed unless there is a major
emergency or complication.} For other deliverables, extensions must be
requested at least 1 business day before the due date. Be prepared to
show appropriate documentation of work to date and to provide an
explanation for the value you will add to the final product if the
extension is granted. Unless a request is made and granted, late
assignments will receive a grade of 0.

\paragraph{Missed quizzes}\label{missed-quizzes}

Quizzes may be rescheduled only in the case of serious illness, injury,
or significant academic or professional development opportunity.
Personal travel or family vacations will not be accommodated unless they
are associated with a significant life event (birth, death, marriage,
graduation). In all cases, documentation is required. Contact me via
email as soon as you know you have a conflict. Local or phone/Skype job
interviews are not considered significant professional development
opportunities.

\subsubsection{Attendance and
participation}\label{attendance-and-participation}

Although attendance will not be formally recorded for all class
meetings, your Team can and should reduce your peer evaluation score if
you miss meetings or are absent during in class Work Days. To avoid this
consequence, you should:

\begin{itemize}
\tightlist
\item
  Communicate clearly with your team any time you will be absent from a
  Team Meeting or class on a day that includes Case Work Time.\\
\item
  Document that communication (e.g.~save emails or post to team message
  board).\\
\item
  Participate actively in scheduling Team Meetings outside of class.\\
\item
  Lay out policies on scheduling team meetings and missed meetings and
  follow them.\\
\item
  Respect others' time and reserve absences for only when absolutely
  unavoidable.\\
\item
  Communicate with me (your instructor) or the Office of Student Life if
  any situation arises that will dramatically impact your attendance in
  class and at team meetings (e.g.~serious illness, injury or personal
  emergency).
\end{itemize}

If you are absent from class, it is your responsibility to obtain notes
and other materials from colleagues. While you are welcome to book
student hours, you should not view them as a substitute for coming to
class. If you are absent regularly without a documented extenuating
circumstance, requests for student hours meetings may be denied.

\subsubsection{Expectations for graded
work}\label{expectations-for-graded-work}

\paragraph{Individual assignments}\label{individual-assignments-1}

I provide students feedback and/or scores on assignments that require
individualized grading before a further assignment of a similar format
is due or within 21 days of submission, whichever is shorter. This
includes quizzes. I will notify you if I am unable to complete grading
within the 21-day timeframe, and will identify a revised return date. If
you submit work after the due date, it may not be returned within 21
days, and depending on the justification for late submission, it may not
be accepted at all. Your graded coursework will be returned in
compliance with FERPA regulations, such as in class, during my student
hours, or via the course management system through which only you will
have access to your grades.

\subsubsection{Grading scale}\label{grading-scale}

\begin{longtable*}[t]{ll}
\toprule
Letter Grade & Percentage Equivalent\\
\midrule
A & 93-100\%\\
AB & 89-92\%\\
B & 83-88\%\\
BC & 79-82\%\\
C & 70-78\%\\
D & 60-69\%\\
F & 59\% - below\\
\bottomrule
\end{longtable*}

\subsection{AI usage limitations}\label{ai-usage-limitations}

Artificial intelligence (AI) tools and resources used for creating text,
images, computer code, audio, or other media, may be used in certain
circumstances in this class. The specifics of when, where, and how these
tools are permitted will be included with each assignment. Guidance for
attribution/citation will be provided by your instructor. Any use of
these tools other than where/how indicated is a violation of course
expectations and is subject to UW-La Crosse's
\href{https://www.uwlax.edu/student-life/our-services/student-conduct/academic-misconduct/}{Academic
Misconduct Policy}. Students are responsible for verifying the accuracy
and appropriateness of content composed by AI. If you are in doubt as to
whether you are using a tool inappropriately in this course, you are
encouraged to discuss your situation with your instructor.

\subsection{Regular and substantive
interaction}\label{regular-and-substantive-interaction}

In this course, regular and substantive interaction between the
instructor and students plays a crucial role in student achievement. I
will be incorporating various forms of interaction that not only meet
the requirements set by the US Department of Education and the UWL
Faculty Senate policy, but also aid in student achievement. Your active
participation and contributions are also essential to your learning. In
partnership with your efforts, I will\ldots{}

\begin{itemize}
\tightlist
\item
  be available during scheduled student hours as stated in the
  syllabus\\
\item
  share information about the course materials including key
  information, explanations, examples, and resources via in-person,
  and/or recorded lectures\\
\item
  engage in online discussions about course content\\
\item
  provide group or individual feedback on assignments\\
\item
  promptly respond to questions about the course sent via email\\
\item
  regularly post announcements pertaining to the course content and
  activities\\
\item
  monitor your academic progress and communicate concerns, as needed
\end{itemize}

\subsection{UWL syllabus policy information \&
statements}\label{uwl-syllabus-policy-information-statements}

UWL encourages students to know the campus' important policies and
statements which can be found on the Syllabus Information website.

\begin{itemize}
\tightlist
\item
  \href{https://www.uwlax.edu/info/syllabus/\#tm-261340}{Classes during
  inclement weather}
\item
  \href{https://www.uwlax.edu/info/syllabus/\#tm-111043}{Religious
  accommodations}
\item
  \href{https://www.uwlax.edu/info/syllabus/\#tm-111041}{Sexual
  misconduct}
\item
  \href{https://www.uwlax.edu/info/syllabus/\#tm-111060}{Student course
  and faculty-related concerns, complaints, and grievances}
\item
  \href{https://www.uwlax.edu/info/syllabus/\#tm-264115}{Student survey
  on instruction (LENS)}
\item
  \href{https://www.uwlax.edu/info/syllabus/\#tm-111053}{Students with
  accommodation needs}
\item
  \href{https://www.uwlax.edu/info/syllabus/\#tm-264137}{University
  class attendance policy}
\item
  \href{https://www.uwlax.edu/info/syllabus/\#tm-111056}{Veterans,
  active military, and military-connected}
\end{itemize}

UWL encourages students to know the campus' important policies and
statements which can be found on the
\href{https://www.uwlax.edu/info/syllabus/}{UWL Syllabus Policy
Information \& Statements}.

\subsection{UWL policies \& supports}\label{uwl-policies-supports}

\subsubsection{Course access}\label{course-access}

Access to course materials in Canvas may cease after the term ends. If
you wish to archive materials for your personal records or portfolio you
should do so as you progress through the course. As a general rule, you
should always save local copies of course-related work. To avoid
disasters, you should also save important files to external media or
cloud storage.

\subsubsection{Inclusive excellence}\label{inclusive-excellence}

\href{https://www.uwlax.edu/chancellor/mission/}{Mission} ``Diversity,
equity, and the inclusion and engagement of all people in a safe campus
climate that embraces and respects the innumerable different
perspectives found within an increasingly integrated and culturally
diverse global community''. If you are not experiencing my class in this
manner, please come talk to me about your experiences so I can try to
adjust the course if possible.

\subsubsection{Name/pronouns}\label{namepronouns}

I will do my best to address you by a preferred name or gender pronoun
that you have identified. Please advise me of this preference early in
the semester so that I may make appropriate changes to my records. UWL
has \href{https://www.uwlax.edu/records/name-in-use/}{name in use
resources} and the \href{https://www.uwlax.edu/pride-center/}{Pride
Center} is available for additional assistance.

\subsubsection{Academic success and overall
health}\label{academic-success-and-overall-health}

At UWL, we support your academic success and overall health. We know
that students often experience a range of stressors that can impact
learning and well-being. If you or someone you know is experiencing
mental health concerns, or could benefit from effective academic
strategies, there are free and confidential resources available to
enrolled students through the Counseling \& Testing Center (CTC). To
learn more, visit the
\href{https://www.uwlax.edu/counseling-testing/}{CTC Website} or call
608-785-8073.

\subsubsection{Technical support}\label{technical-support}

For tips and information about Canvas visit the (canvas
website){[}https://www.uwlax.edu/info/canvas/students/{]}; this site
also links to the 24/7 Canvas support. If you are having Canvas login
issues or need general computer assistance, contact the (Eagle Help
Desk){[}https://www.uwlax.edu/its/client-services-and-support/eagle-help-desk/{]}.

\subsubsection{Student success
resources}\label{student-success-resources}

\textbf{UWL also encourages students to take advantage of the campus'
many and varied (Campus Life
Resources){[}https://www.uwlax.edu/info/campus-life/{]}.}

\subsection{Course outline}\label{course-outline}

\begin{longtable*}[t]{ll}
\toprule
Module Name & Module Learning Outcomes\\
\midrule
Baseline Data Literacy & Describe the anatomy of a data set including type (primary process generated or administrative, primary purpose generated, secondary), grouping, and unit of analysis.\\
Baseline Data Literacy & Distinguish between discrete and continuous variables, and between nominal, ordinal and ratio scale variables.\\
Baseline Data Literacy & Compute and interpret descriptive statistics relevant to the purpose of the analysis and appropriate for the level of measurement of the variables.\\
Baseline Data Literacy & Interpret common visual representations of data including bar charts, histograms, and scatter plots.\\
Baseline Data Literacy & Organize data in a spreadsheet following best practices.\\
Baseline Data Literacy & Create common visual representations of data including bar charts, histograms, and scatter plots.\\
Baseline Data Literacy & Demonstrate basic software competencies in spreadsheet software.\\
Baseline Data Literacy & Define and appropriately use common terms in data analysis including correlation, causation, generalizability, validity, and reliability.\\
Foundations of Data Analysis (Course LO) & Apply appropriate univariate and bivariate inferential statistics and correctly interpret the results of statistical tests.\\
Foundations of Data Analysis (Course LO) & Demonstrate basic software competencies in scripting software.\\
Communication of Results & Curate results generated to identify the most relevant insights and supporting analyses.\\
Communication of Results & Create clear and compelling visualizations with necessary annotation to achieve stand alone effectiveness.\\
Communication of Results & Summarize the problem addressed, methods used, results, and implications clearly and concisely in written and oral communication formats intended for non-technical audiences.\\
Communication of Results & Quantify the implications of alternative solutions or proposed recommendations that follow from analysis.\\
Communication of Results & Document methods and detailed results of analysis in appendices intended for technical audiences.\\
Communication of Results & Compose succinct professional emails communicating progress on a project.\\
Problem Definition & Gather insights from existing reports or others' analyses to demonstrate and articulate the existence of a business problem or opportunity.\\
Problem Definition & Define research objectives that are feasible and hypothesis that can be tested using the chosen study methods and data sources.\\
Problem Definition & Select relevant variables and sources of data needed to analyze an identified business problem or opportunity and propose survey measures when required data do not exist.\\
Problem Definition & Define key metrics.\\
Problem Definition & Choose appropriate analysis methods, data, and variables for analyzing the identified problem or opportunity.\\
Intro to Advanced Topics & Distinguish between observational, quasi-experimental, and observational study designs and select the appropriate design for research objectives and desired use of findings.\\
Intro to Advanced Topics & Define advanced analysis methods such as regression, forecasting, quasi-experimental and experimental study designs, optimization, machine learning, and text mining, and identify appropriate use cases.\\
Intro to Advanced Topics & Prepare clear, appropriate, and appealing visual summaries of data in graphs and tables with informative use of color and text.\\
Intro to Advanced Topics & Prepare data for analysis using sorting, filtering, merging, and additional functions as appropriate.\\
Intro to Advanced Topics & Design a survey instrument to collect data needed to address a research question, identify potential sources of bias and articulate approach to avoiding bias in the design.\\
Becoming an Informed Consumer of Data & Suggest relevant alternative explanations for results.\\
Becoming an Informed Consumer of Data & Correct errors in analysis and interpretation of analysis.\\
Becoming an Informed Consumer of Data & Describe current debates in research ethics and data privacy and the implications for business research.\\
\bottomrule
\end{longtable*}

\subsection{Important dates}\label{important-dates}

\begin{longtable*}[t]{lll}
\toprule
Name & Date & Event\\
\midrule
Week 3 & 2/10/2026 & Graded Quiz: Descriptive Stats \& Data Literacy\\
Week 5 & 2/24/2026 & Graded Quiz: Data Visualization\\
Week 7 & 3/10/2026 & Graded Quiz: R Basics\\
Week 9 & 3/31/2026 & Presentation Coaching\\
Week 11 & 4/7/2026 & Graded Quiz: Inferential Stats \& Data Literacy\\
Week 12 & 4/14/2026 & Analysis Project Final Presentations\\
Week 13 & 4/28/2026 & Graded Quiz: Survey Research Methods\\
Week 14 & 5/5/2026 & Research Proposal Presentations\\
Final Exam & Tuesday 05/12 07:00-09:00PM & Final Exam \& Practicum\\
\bottomrule
\end{longtable*}

\subsection{Anticipated schedule}\label{anticipated-schedule}

When discrepancies between Canvas and the Anticipated Schedule exist,
the information on Canvas supersedes this schedule. Official class
agendas are posted under This Week in Canvas.

\begin{longtable*}[t]{lllll}
\toprule
Class & Date & Part 1 & Part 2 & Part 3\\
\midrule
Week 1 & 1/27/2026 & Course Overview & Anatomy, Grouping \& Unit of Analysis & Data Explore in 3 softwares\\
Week 2 & 2/3/2026 & Level of Measurement Descriptive Statistics & Measures of Variablity and Association & Data Explore in 3 softwares\\
Week 3 & 2/10/2026 & Data Visualization & Data Visualization in Tableau & Data Visualization in Tableau\\
Week 4 & 2/17/2026 & Communicating with Data in Writing & Communicating with Data Orally & Work Time\\
Week 5 & 2/24/2026 & Problem Definition & Launch Analysis Project & Work Time\\
Week 6 & 3/3/2026 & Inference & Rstudio Descriptive Statistics & Work Time\\
Week 7 & 3/10/2026 & Hypothesis Testing & Rstudio Inferential Statistics & Work Time\\
Week 8 & 3/24/2026 & Pre-Coaching Work & Work Time & Work Time\\
Week 9 & 3/31/2026 & Coaching Sessions & Work Time & Work Time\\
Week 10 & 4/7/2026 & Communicating with Data Documenting Work & Work Time & Work Time\\
Week 11 & 4/14/2026 & Analysis Project Final Presentations & Analysis Project Final Presentations & The Research Proposal\\
Week 12 & 4/21/2026 & Advanced Analytics: Sample Design & Advanced Analytics: Survey Design & \\
Week 13 & 4/28/2026 & Advanced Analytics: Machine Learning & Advanced Analytics: Experiment Design & Ethics\\
Week 14 & 5/5/2026 & Research Proposal Presentations & Research Proposal Presentations & Final Exam Prep\\
Final Exam & 5/12/2026 & Final Exam Tuesday 05/12 07:00-09:00PM &  & \\
\bottomrule
\end{longtable*}




\end{document}
